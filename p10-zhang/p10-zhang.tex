% This is a template for producing artifact descriptions associated with ECOOP 2017 papers
%
% Following is the notice from Camil Demetrescu's ECOOP 2016 template on which this
% template is based:
% This is a template for producing artifact descriptions associated with ECOOP papers
% Adapted from the standard LIPIcs template by Camil Demetrescu
% See lipics-manual.pdf for further information.
% April 22, 2015

\documentclass[a4paper,UKenglish]{darts}
% for A4 paper format use option "a4paper", for US-letter use option "letterpaper"
% for british hyphenation rules use option "UKenglish", for american hyphenation rules use option "USenglish"
% for section-numbered lemmas etc., use "numberwithinsect"
 
\usepackage{microtype}%if unwanted, comment out or use option "draft"

%\graphicspath{{./graphics/}}%helpful if your graphic files are in another directory

\bibliographystyle{plainurl}% the recommended bibstyle

% ARTIFACT
\newenvironment{scope}{\section{Scope}}{}
\newenvironment{content}{\section{Content}}{}
\newenvironment{getting}{\section{Getting the artifact} The artifact 
endorsed by the Artifact Evaluation Committee is available free of 
charge on the Dagstuhl Research Online Publication Server (DROPS).}{}
\newenvironment{platforms}{\section{Tested platforms}}{}
\newcommand{\license}[1]{{\section{License}#1}}
\newcommand{\mdsum}[1]{{\section{MD5 sum of the artifact}#1}}
\newcommand{\artifactsize}[1]{{\section{Size of the artifact}#1}}

% Author macros::begin %%%%%%%%%%%%%%%%%%%%%%%%%%%%%%%%%%%%%%%%%%%%%%%%
% ARTIFACT: Please use the same title as the corresponding ECOOP paper and append the text "(Artifact)"
% ARTIFACT: Add as a footnote the reference to the corresponding ECOOP paper
\title{EVF: An Extensible and Expressive Visitor Framework for Programming
  Language Reuse (Artifact)\footnote{This artifact is a companion of the paper:
    Weixin Zhang and Bruno C. d. S. Oliveira, ``EVF: An Extensible and Expressive Visitor Framework for Programming
  Language Reuse'', Proceedings of the 31st European Conference on Object-Oriented Programming (ECOOP 2017), June 18-23, 2017, Barcelona, Spain.}}
\titlerunning{EVF: An Extensible and Expressive Visitor Framework for Programming
  Language Reuse (Artifact)} %optional, in case that the title is too long; the running title should fit into the top page column

% ARTIFACT: Authors may not be exactly the same as the ECOOP paper, e.g., you may want to include authors who contributed to the preparation of the artifact, but not to the ECOOP companion paper
%% Please provide for each author the \author and \affil macro, even when authors have the same affiliation, i.e. for each author there needs to be the  \author and \affil macros
\author{Weixin Zhang}
\author{Bruno C. d. S. Oliveira}
\affil{The University of Hong Kong, Hong Kong, China\\
  \texttt{\{wxzhang2,bruno\}@cs.hku.hk}}
%\affil[2]{The University of Hong Kong, Hong Kong, China\\
%  \texttt{bruno@cs.hku.hk}}
\authorrunning{W. Zhang and B.\,C.\,d.\,S. Oliveira} %mandatory. First: Use abbreviated first/middle names. Second (only in severe cases): Use first author plus 'et. al.'

\Copyright{Weixin Zhang and Bruno C. d. S. Oliveira}%mandatory, please use full first names. LIPIcs license is "CC-BY";  http://creativecommons.org/licenses/by/3.0/

\subjclass{D.1.5 Object-oriented Programming, D.3.3 Language Constructs and Features, D.3.4 Processors}% mandatory: Please choose ACM 1998 classifications from http://www.acm.org/about/class/ccs98-html . E.g., cite as "F.1.1 Models of Computation".
\keywords{Visitor Pattern, Object Algebras, Modularity, Domain-Specific Languages}% mandatory: Please provide 1-5 keywords
% Author macros::end %%%%%%%%%%%%%%%%%%%%%%%%%%%%%%%%%%%%%%%%%%%%%%%%%

%Editor-only macros:: begin (do not touch as author)%%%%%%%%%%%%%%%%%%%%%%%%%%%%%%%%%%
\Volume{3}
\Issue{2}
\Article{10}
\RelatedConference{European Conference on Object-Oriented Programming (ECOOP 2017), June 18-23, 2017, Barcelona, Spain}
% Editor-only macros::end %%%%%%%%%%%%%%%%%%%%%%%%%%%%%%%%%%%%%%%%%%%%%%%

\begin{document}

\maketitle

\begin{abstract}
  This artifact is based on {\bf EVF}, an extensible and expressive Java
  {\sc visitor} framework. {\bf EVF} aims at reducing the effort involved in
  creation and reuse of programming languages. {\bf EVF} an annotation processor
  that automatically generate boilerplate ASTs and AST for a given an Object
  Algebra interface. This artifact contains source code of the
  case study on ``Types and Programming Languages'', illustrating how effective
  {\bf EVF} is in modularizing programming languages that
  There is also a microbenchmark in the artifact that shows that {\bf EVF} has
  reasonable performance with respect to traditional visitors.
 \end{abstract}

% ARTIFACT: please stick to the structure of 7 sections provided below

% ARTIFACT: section on the scope of the artifact (what claims of the paper are intended to be backed by this artifact?)
\begin{scope}
  The artifact is designed to support repeatability of all the experiments of the 
  companion paper, allowing users to test the framework on a variety of benchmarks.
  It includes {\bf EVF}, an extensible and expressive Java {\sc Visitor} framework
  that aims at reducing the effort involved in
  creating and reusing programming languages.
  {\bf EVF} is best used with an IDE like Eclipse, which automatically generates
  boilerplate ASTs and AST traversals for an annotated standard Object Algebra
  interface by saving.
  This artifact also contains the source code of the case study on ``Types and
  Programming Languages'' and the microbenchmark that we discussed in the
  companion paper.
\end{scope}

% ARTIFACT: section on the contents of the artifact (code, data, etc.)
\begin{content}
  The artifact package includes:
  \begin{itemize}
  \item \emph{VisitProcessor}: the Java annotation processor;
  \item \emph{tapl}: the case study on ``Types and programming languages'';
  \item \emph{benchmark}: the microbenchmark.
  \end{itemize}
  Detailed instructions for using the artifact and for rebuilding it from scratch, provided as an {\tt README.md} file.
\end{content}

% ARTIFACT: section containing links to sites holding the
% latest version of the code/data, if any
\begin{getting}
% leave empty if the artifact is only available on the DROPS server.
% otherwise, provide links to the latest version of the artifact (e.g., on github)
  The latest version of our code is available on GitHub:
  \url{https://github.com/wxzh/EVF}.
\end{getting}

% ARTIFACT: section specifying the platforms on which the artifact is known to
% work, including requirements beyond the operating system such as large
% amounts of memory or many processor cores
\begin{platforms}
  The artifact is known to work on any platform running JDK (version 1.8 or later)
  with an Eclipse IDE (version 4.5.1 or later). To run the scripts, the artifact additionally requires ruby
  (version 2.0.0 or later) and cloc (version 1.62 or later) installed.
\end{platforms}

% ARTIFACT: section specifying the license under which the artifact is
% made available
\license{BSD}

% ARTIFACT: section specifying the md5 sum of the artifact master file
% uploaded to the Dagstuhl Research Online Publication Server, enabling 
% downloaders to check that the file is the expected version and suffered 
% no damage during download.
\mdsum{afe518a203489ef8fc1f7c93435dd35e}

% ARTIFACT: section specifying the size of the artifact master file uploaded
% to the Dagstuhl Research Online Publication Server
\artifactsize{698KB}

\subparagraph*{Acknowledgements}

We would like to thank the anonymous reviewers for their helpful comments and suggestions. This work has been sponsored by the Hong Kong Research Grant Council projects number 27200514 and 17258816.

% ARTIFACT: optional appendix
%\appendix

%\section{My Appendix}

% Add here any further material you would like to include. For instance, if the artifact is itself a PDF document, add it here.


% ARTIFACT: include here any additional references, if needed...

%% Either use bibtex (recommended), but commented out in this sample

%\bibliography{dummybib}

%% .. or use the thebibliography environment explicitely


\end{document}
