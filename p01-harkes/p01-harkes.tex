% This is a template for producing artifact descriptions associated with ECOOP 2017 papers
%
% Following is the notice from Camil Demetrescu's ECOOP 2016 template on which this
% template is based:
% This is a template for producing artifact descriptions associated with ECOOP papers
% Adapted from the standard LIPIcs template by Camil Demetrescu
% See lipics-manual.pdf for further information.
% April 22, 2015

\documentclass[a4paper,USenglish]{darts}
% for A4 paper format use option "a4paper", for US-letter use option "letterpaper"
% for british hyphenation rules use option "UKenglish", for american hyphenation rules use option "USenglish"
% for section-numbered lemmas etc., use "numberwithinsect"
 
\usepackage{microtype}%if unwanted, comment out or use option "draft"

%\graphicspath{{./graphics/}}%helpful if your graphic files are in another directory

\bibliographystyle{plainurl}% the recommended bibstyle

% ARTIFACT
\newenvironment{scope}{\section{Scope}}{}
\newenvironment{content}{\section{Content}}{}
\newenvironment{getting}{\section{Getting the artifact} The artifact 
endorsed by the Artifact Evaluation Committee is available free of 
charge on the Dagstuhl Research Online Publication Server (DROPS).}{}
\newenvironment{platforms}{\section{Tested platforms}}{}
\newcommand{\license}[1]{{\section{License}#1}}
\newcommand{\mdsum}[1]{{\section{MD5 sum of the artifact}#1}}
\newcommand{\artifactsize}[1]{{\section{Size of the artifact}#1}}

% Author macros::begin %%%%%%%%%%%%%%%%%%%%%%%%%%%%%%%%%%%%%%%%%%%%%%%%
% ARTIFACT: Please use the same title as the corresponding ECOOP paper and append the text "(Artifact)"
% ARTIFACT: Add as a footnote the reference to the corresponding ECOOP paper
\title{IceDust 2: Derived Bidirectional Relations and Calculation Strategy Composition (Artifact)\footnote{This artifact is a companion of the paper:  Daco C. Harkes and Eelco Visser, ``IceDust 2: Derived Bidirectional Relations and Calculation Strategy Composition'', Proceedings of the 31st European Conference on Object-Oriented Programming (ECOOP 2017), June 18-23, 2017, Barcelona, Spain. This research was funded by the NWO VICI \textit{Language Designer’s Workbench} project (639.023.206).}}

\titlerunning{IceDust 2: Derived Relations and Calculation Strategy Composition (Artifact)} %optional, in case that the title is too long; the running title should fit into the top page column

% ARTIFACT: Authors may not be exactly the same as the ECOOP paper, e.g., you may want to include authors who contributed to the preparation of the artifact, but not to the ECOOP companion paper
%% Please provide for each author the \author and \affil macro, even when authors have the same affiliation, i.e. for each author there needs to be the  \author and \affil macros

\author[1]{Daco C. Harkes\footnote{Core artifact developer.}}
\author[2]{Eelco Visser}
\affil[1]{Delft University of Technology, Delft, The Netherlands\\
  \texttt{d.c.harkes@tudelft.nl}}
\affil[2]{Delft University of Technology, Delft, The Netherlands\\
  \texttt{e.visser@tudelft.nl}}
  
\authorrunning{D.\,C. Harkes, and E. Visser} %mandatory. First: Use abbreviated first/middle names. Second (only in severe cases): Use first author plus 'et. al.'

\Copyright{Daco C. Harkes, and Eelco Visser}%mandatory, please use full first names. LIPIcs license is "CC-BY";  http://creativecommons.org/licenses/by/3.0/

\subjclass{D.3.2 Data-flow languages}% mandatory: Please choose ACM 1998 classifications from http://www.acm.org/about/class/ccs98-html . E.g., cite as "F.1.1 Models of Computation" -- ARTIFACT: You may use the same as the corresponding ECOOP paper.

\keywords{Incremental Computing, Data Modeling, Domain Specific Language}% mandatory: Please provide 1-5 keywords -- ARTIFACT: You may use the same as the corresponding ECOOP paper.
% Author macros::end %%%%%%%%%%%%%%%%%%%%%%%%%%%%%%%%%%%%%%%%%%%%%%%%%

%Editor-only macros:: begin (do not touch as author)%%%%%%%%%%%%%%%%%%%%%%%%%%%%%%%%%%
\Volume{3}
\Issue{2}
\Article{1}
\RelatedConference{European Conference on Object-Oriented Programming (ECOOP 2017), June 18-23, 2017, Barcelona, Spain}
% Editor-only macros::end %%%%%%%%%%%%%%%%%%%%%%%%%%%%%%%%%%%%%%%%%%%%%%%

\begin{document}

\maketitle

\begin{abstract}
  This artifact is based on {\tt IceDust2}, a data modeling language with derived values.
  The provided package is designed to support the claims of the companion paper: in particular, it allows users to compile and run IceDust2 specifications.
  Instructions for building the IceDust2 compiler from source in Spoofax are also provided.
\end{abstract}

% ARTIFACT: please stick to the structure of 7 sections provided below

% ARTIFACT: section on the scope of the artifact (what claims of the paper are intended to be backed by this artifact?)
\begin{scope}
  The provided package is designed to support the following claims of the companion paper:
  \begin{itemize}
    \item IceDust2 has two working back ends.
    \item We did two case studies.
  \end{itemize}  
\end{scope}

% ARTIFACT: section on the contents of the artifact (code, data, etc.)
\begin{content}
  The archive contains the following file and folders:
  \begin{itemize}
  \item A description on how to build the IceDust2 compiler and how to comiple IceDust2 files.
  \item The source files for the IceDust compiler, a test suite, and IceDust example programs.
  \item The language workbench Spoofax (in Eclipse) in which IceDust is developed. The folder contains a Linux, Windows and MacOS version.
  \item The WebDSL compiler (also in Eclipse), which IceDust compiles to for persistence and concurrency. The folder contains a Linux, Windows and MacOS version.
  \item Two empty folders, suggested to use as workspace for Spoofax and WebDSL.
  \end{itemize}
\end{content} 

% ARTIFACT: section containing links to sites holding the
% latest version of the code/data, if any
\begin{getting}
% leave empty if the artifact is only available on the DROPS server.
% otherwise, provide links to the latest version of the artifact (e.g., on github)
  The latest version of our code is available on GitHub: {\url {https://github.com/MetaBorgCube/IceDust}}.
\end{getting} 

% ARTIFACT: section specifying the platforms on which the artifact is known to
% work, including requirements beyond the operating system such as large
% amounts of memory or many processor cores
\begin{platforms}
  The artifact is known to work on macOS Sierra and Windows 10 with at least 5~GB of free
  space on disk and at least 4~GB of free space in RAM.
  JDK 8 or newer needs to be available on your path. (Both {\tt java} and {\tt javac} should be available in the terminal.)
\end{platforms}

% ARTIFACT: section specifying the license under which the artifact is
% made available
\license{EPL-1.0 ({\url {http://www.eclipse.org/legal/epl-v10.html}})}

% ARTIFACT: section specifying the md5 sum of the artifact master file
% uploaded to the Dagstuhl Research Online Publication Server, enabling 
% downloaders to check that the file is the expected version and suffered 
% no damage during download.
\mdsum{65f4eb739bd175221e1e4962b9dad5b8}

% ARTIFACT: section specifying the size of the artifact master file uploaded
% to the Dagstuhl Research Online Publication Server
\artifactsize{2.1 GB}



% ARTIFACT: optional appendix
%\appendix

%\section{My Appendix}

% Add here any further material you would like to include. For instance, if the artifact is itself a PDF document, add it here.


% ARTIFACT: include here any additional references, if needed...

%% Either use bibtex (recommended), but commented out in this sample

%\bibliography{dummybib}

%% .. or use the thebibliography environment explicitely




\end{document}
