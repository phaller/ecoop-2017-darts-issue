% This is a template for producing artifact descriptions associated with ECOOP 2017 papers
%
% Following is the notice from Camil Demetrescu's ECOOP 2016 template on which this
% template is based:
% This is a template for producing artifact descriptions associated with ECOOP papers
% Adapted from the standard LIPIcs template by Camil Demetrescu
% See lipics-manual.pdf for further information.
% April 22, 2015

\documentclass[letterpaper,USenglish]{darts}
% for A4 paper format use option "a4paper", for US-letter use option "letterpaper"
% for british hyphenation rules use option "UKenglish", for american hyphenation rules use option "USenglish"
% for section-numbered lemmas etc., use "numberwithinsect"
 
%\usepackage{microtype}%if unwanted, comment out or use option "draft"

%\graphicspath{{./graphics/}}%helpful if your graphic files are in another directory

\bibliographystyle{plainurl}% the recommended bibstyle

% ARTIFACT
\newenvironment{scope}{\section{Scope}}{}
\newenvironment{content}{\section{Content}}{}
\newenvironment{getting}{\section{Getting the artifact} The artifact 
endorsed by the Artifact Evaluation Committee is available free of 
charge on the Dagstuhl Research Online Publication Server (DROPS).}{}
\newenvironment{platforms}{\section{Tested platforms}}{}
\newcommand{\license}[1]{{\section{License}#1}}
\newcommand{\mdsum}[1]{{\section{MD5 sum of the artifact}#1}}
\newcommand{\artifactsize}[1]{{\section{Size of the artifact}#1}}

% Author macros::begin %%%%%%%%%%%%%%%%%%%%%%%%%%%%%%%%%%%%%%%%%%%%%%%%
% ARTIFACT: Please use the same title as the corresponding ECOOP paper and append the text "(Artifact)"
% ARTIFACT: Add as a footnote the reference to the corresponding ECOOP paper
\title{Parallelizing Julia with a Non-invasive DSL (Artifact)\footnote{This artifact is a companion of the paper:  Todd A. Anderson, Hai Liu, Lindsey Kuper, Ehsan Totoni, Jan Vitek and Tatiana Shpeisman, ``Parallelizing Julia with a Non-invasive DSL'', Proceedings of the 31st European Conference on Object-Oriented Programming (ECOOP 2017), June 18-23, 2017, Barcelona, Spain.}}
\titlerunning{Parallelizing Julia with a Non-invasive DSL (Artifact)} %optional, in case that the title is too long; the running title should fit into the top page column

% ARTIFACT: Authors may not be exactly the same as the ECOOP paper, e.g., you may want to include authors who contributed to the preparation of the artifact, but not to the ECOOP companion paper
%% Please provide for each author the \author and \affil macro, even when authors have the same affiliation, i.e. for each author there needs to be the  \author and \affil macros
\author[1]{Todd A. Anderson}
\author[1]{Hai Liu}
\author[1]{Lindsey Kuper}
\author[1]{Ehsan Totoni}
\author[2]{Jan Vitek}
\author[1]{Tatiana Shpeisman}
\affil[1]{Parallel Computing Lab, Intel Labs}
\affil[2]{Northeastern University / Czech Technical University Prague}
\authorrunning{T. A. Anderson, H. Liu, L. Kuper, E. Totoni, J. Vitek, and T. Shpeisman} 

\Copyright{Todd A. Anderson, Hai Liu, Lindsey Kuper, Ehsan Totoni, Jan Vitek, and Tatiana Shpeisman}
%mandatory, please use full first names. LIPIcs license is
            %"CC-BY"; http://creativecommons.org/licenses/by/3.0/
\subjclass{D.1.3 Parallel Programming}

\keywords{parallelism, scientific computing, domain-specific languages, Julia}

%Editor-only macros:: begin (do not touch as author)%%%%%%%%%%%%%%%%%%%%%%%%%%%%%%%%%%
\Volume{3}
\Issue{2}
\Article{7}
\RelatedConference{European Conference on Object-Oriented Programming (ECOOP 2017), June 18-23, 2017, Barcelona, Spain}
% Editor-only macros::end %%%%%%%%%%%%%%%%%%%%%%%%%%%%%%%%%%%%%%%%%%%%%%%

\begin{document}

\maketitle

\begin{abstract}
  This artifact is based on ParallelAccelerator, an embedded
  domain-specific language (DSL) and compiler for speeding up
  compute-intensive Julia programs.  In particular, Julia code that
  makes heavy use of aggregate array operations is a good candidate
  for speeding up with ParallelAccelerator.  ParallelAccelerator is a
  \emph{non-invasive} DSL that makes as few changes to the host
  programming model as possible.
\end{abstract}

% ARTIFACT: please stick to the structure of 7 sections provided below

% ARTIFACT: section on the scope of the artifact (what claims of the paper are intended to be backed by this artifact?)
\begin{scope}
  This artifact allows the user to install, run, test, and benchmark
  the ParallelAccelerator Julia package.  It is designed to support
  repeatability of the experiments of the companion paper.
\end{scope}

% ARTIFACT: section on the contents of the artifact (code, data, etc.)
\begin{content}
  This artifact contains:

  \begin{itemize}
    \item Snapshots of the most recent tagged releases of the
      ParallelAccelerator (v0.2.2) and CompilerTools (v0.2.1) Julia
      packages at the time of artifact release.  In addition to source
      code for the ParallelAccelerator compiler itself, the
      ParallelAccelerator package also includes the code for
      implementations (using both ParallelAccelerator and plain Julia)
      of all the workloads discussed in the companion paper, as well
      as for a few additional workloads.
    \item Additional MATLAB, Python, and C/C++ implementations of the
      workloads discussed in the companion paper, as well as code
      for benchmarking and plotting of benchmarking results.
    \item Detailed instructions for using the artifact, provided as a
      {\tt README.pdf} file.
  \end{itemize}
\end{content} 

% ARTIFACT: section containing links to sites holding the
% latest version of the code/data, if any
\begin{getting}
% leave empty if the artifact is only available on the DROPS server.
% otherwise, provide links to the latest version of the artifact (e.g., on github)
  The latest version of ParallelAccelerator is available on GitHub at
  \url{https://github.com/IntelLabs/ParallelAccelerator.jl}.
\end{getting} 

% ARTIFACT: section specifying the platforms on which the artifact is known to
% work, including requirements beyond the operating system such as large
% amounts of memory or many processor cores
\begin{platforms}
  ParallelAccelerator requires a *nix OS, ideally Linux.  Platforms we
  have tested on include Ubuntu 16.04, Ubuntu 14.04, CentOS 6.6, macOS
  Yosemite, and macOS Sierra.
\end{platforms}

% ARTIFACT: section specifying the license under which the artifact is
% made available
\license{See
  \url{https://github.com/IntelLabs/ParallelAccelerator.jl/blob/master/LICENSE.md}
  for the ParallelAccelerator license.}

% ARTIFACT: section specifying the md5 sum of the artifact master file
% uploaded to the Dagstuhl Research Online Publication Server, enabling 
% downloaders to check that the file is the expected version and suffered 
% no damage during download.
\mdsum{c8eba9e27a8c6c2b45612c883e20bbac}

% ARTIFACT: section specifying the size of the artifact master file uploaded
% to the Dagstuhl Research Online Publication Server
\artifactsize{54.97 MB}

\subparagraph*{Acknowledgements}

We thank Gabriel Scherer for testing, and the ECOOP '17 Artifact
Evaluation Committee reviewers for their helpful comments.

\end{document}
