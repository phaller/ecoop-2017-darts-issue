% This is a template for producing artifact descriptions associated with ECOOP 2017 papers
%
% Following is the notice from Camil Demetrescu's ECOOP 2016 template on which this
% template is based:
% This is a template for producing artifact descriptions associated with ECOOP papers
% Adapted from the standard LIPIcs template by Camil Demetrescu
% See lipics-manual.pdf for further information.
% April 22, 2015

\documentclass[a4paper,UKenglish]{darts}
% for A4 paper format use option "a4paper", for US-letter use option "letterpaper"
% for british hyphenation rules use option "UKenglish", for american hyphenation rules use option "USenglish"
% for section-numbered lemmas etc., use "numberwithinsect"
 
\usepackage{microtype}%if unwanted, comment out or use option "draft"

%\graphicspath{{./graphics/}}%helpful if your graphic files are in another directory

\bibliographystyle{plainurl}% the recommended bibstyle

% ARTIFACT: Include the following input command here
\newenvironment{scope}{\section{Scope}}{}
\newenvironment{content}{\section{Content}}{}
\newenvironment{getting}{\section{Getting the artifact} The artifact 
endorsed by the Artifact Evaluation Committee is available free of 
charge on the Dagstuhl Research Online Publication Server (DROPS).}{}
\newenvironment{platforms}{\section{Tested platforms}}{}
\newcommand{\license}[1]{{\section{License}#1}}
\newcommand{\mdsum}[1]{{\section{MD5 sum of the artifact}#1}}
\newcommand{\artifactsize}[1]{{\section{Size of the artifact}#1}}

% Author macros::begin %%%%%%%%%%%%%%%%%%%%%%%%%%%%%%%%%%%%%%%%%%%%%%%%
% ARTIFACT: Please use the same title as the corresponding ECOOP paper and append the text "(Artifact)"
% ARTIFACT: Add as a footnote the reference to the corresponding ECOOP paper
\title{Strong Logic for Weak Memory: Reasoning About Release-Acquire Consistency
  in Iris (Artifact)\footnote{This artifact is a companion of the paper: 
  Jan-Oliver Kaiser, Hoang-Hai Dang, Derek Dreyer, Ori Lahav, and Viktor Vafeiadis, 
  ``Strong Logic for Weak Memory: Reasoning About Release-Acquire Consistency
  in Iris'', Proceedings of the 31st European Conference on Object-Oriented Programming (ECOOP 2017), June 18-23, 2017, Barcelona, Spain. This work was supported in part by an ERC Consolidator Grant for the project ``RustBelt'', 
  funded under the European Union's Horizon 2020 Framework Programme (grant agreement no. 683289).}}
\titlerunning{Strong Logic for Weak Memory: Reasoning About RA Consistency
  in Iris (Artifact)} %optional, in case that the title is too long; the running title should fit into the top page column

% ARTIFACT: Authors may not be exactly the same as the ECOOP paper, e.g., you may want to include authors who contributed to the preparation of the artifact, but not to the ECOOP companion paper
%% Please provide for each author the \author and \affil macro, even when authors have the same affiliation, i.e. for each author there needs to be the  \author and \affil macros
\author{Jan-Oliver Kaiser}
\author{Hoang-Hai Dang}
\author{Derek Dreyer}
\author{Ori Lahav}
\author{Viktor Vafeiadis}
\affil{MPI-SWS, Saarland Informatics Campus, Saarbr\"ucken, Germany\\
  \texttt{\{janno,haidang,dreyer,orilahav,viktor\}@mpi-sws.org}}

\authorrunning{J.-O. Kaiser, H.-H. Dang, D. Dreyer, O. Lahav, and V. Vafeiadis} %mandatory. First: Use abbreviated first/middle names. Second (only in severe cases): Use first author plus 'et. al.'

\Copyright{Jan-Oliver Kaiser, Hoang-Hai Dang, Derek Dreyer, Ori Lahav, and Viktor Vafeiadis}%mandatory, please use full first names. LIPIcs license is "CC-BY";  http://creativecommons.org/licenses/by/3.0/

\subjclass{F.3.1 Specifying and Verifying and Reasoning about Programs; F.3.2 Semantics of Programming Languages}
\keywords{Weak memory models, Release-Acquire, Concurrency, Separation logic}

% Author macros::end %%%%%%%%%%%%%%%%%%%%%%%%%%%%%%%%%%%%%%%%%%%%%%%%%

%Editor-only macros:: begin (do not touch as author)%%%%%%%%%%%%%%%%%%%%%%%%%%%%%%%%%%
\Volume{3}
\Issue{2}
\Article{15}
\RelatedConference{European Conference on Object-Oriented Programming (ECOOP 2017), June 18-23, 2017, Barcelona, Spain}
% Editor-only macros::end %%%%%%%%%%%%%%%%%%%%%%%%%%%%%%%%%%%%%%%%%%%%%%%

\begin{document}

\maketitle

\begin{abstract}
  This artifact provides the soundness proofs for the encodings in Iris the RSL 
  and GPS logics, as well as the verification for all standard examples known to
  be verifiable in those logics. All of these proofs are formalized in Coq, which
  is the main content of this artifact. The formalization is provided in a 
  virtual machine for the convenience of testing, but can also be built from source.
 \end{abstract}

% ARTIFACT: please stick to the structure of 7 sections provided below

% ARTIFACT: section on the scope of the artifact (what claims of the paper are intended to be backed by this artifact?)
\begin{scope}
  The artifact is designed to support repeatability of all the proofs of the 
  companion paper, allowing users to step through the proofs in their favorite Coq
  editor. The mapping between the paper and the Coq development is provided
  in the {\tt README.md} file.
\end{scope}

% ARTIFACT: section on the contents of the artifact (code, data, etc.)
\begin{content}
  The artifact package includes a VirtualBox-based Debian virtual machine, which contains
  a copy of the Coq development (revision
  \href{https://gitlab.mpi-sws.org/FP/sra-gps/tree/d7f3799d9750df2754e9b181209cc1a092028724}{d7f3799d9750df2754e9b181209cc1a092028724}). 
  The {\tt README.md} file
  contains a list of lemmas and theorems presented in the paper and where to find
  them in the Coq development. It also contains instructions to build the
  development from source, which requires a system with \textsf{opam} installed.
\end{content} 

% ARTIFACT: section containing links to sites holding the
% latest version of the code/data, if any
\begin{getting}
% leave empty if the artifact is only available on the DROPS server.
% otherwise, provide links to the latest version of the artifact (e.g., on github)
  The latest version of the artifact is available at our project page:
  {\tt {\url http://plv.mpi-sws.org/igps/}}.
\end{getting} 

% ARTIFACT: section specifying the platforms on which the artifact is known to
% work, including requirements beyond the operating system such as large
% amounts of memory or many processor cores
\begin{platforms}
  The artifact is known to work on any platform running Oracle VirtualBox
  version~5.1 with at least 4~GB of free space on disk.

  \paragraph*{Instructions:}
  \begin{enumerate}
  \item (Optional) Increase the number of processors and the amount of memory
    assigned to the VM. This helps with the last step of the build process. (4
    processors and 4GB RAM work reasonably well.)
  \item The machine will boot into a minimal Debian installation. The user and
    password are ``artefact'', as is the password to execute \textsf{su}.
  \item After logging in, open a terminal and navigate to
    \texttt{\textasciitilde/ra-gps} -- this is a copy of the commit mentioned
    above.
  \item Proceed to \texttt{\textasciitilde/ra-gps/coq/ra} and execute \textsf{make
      build-dep}.
    This will install all dependencies of our development. \textbf{Note:} this
    command can take between 30 and 90 minutes. (For reasons unknown to us, this
    step does not use all available processors.)
  \item Finally, execute \textsf{make}, or \textsf{make -jN} where N is the number
    of processors assigned to the VM. This step will terminate in ca. 20 minutes
    with ${\textsf{make~-j4}}$ or around 50 minutes with just \textsf{make}. It should finish
    without errors, indicating that all Coq files were compiled successfully.
  \end{enumerate}

\end{platforms}

% ARTIFACT: section specifying the license under which the artifact is
% made available
\license{BSD}

% ARTIFACT: section specifying the md5 sum of the artifact master file
% uploaded to the Dagstuhl Research Online Publication Server, enabling 
% downloaders to check that the file is the expected version and suffered 
% no damage during download.
\mdsum{dbdf8c863bb606643d8ec203012362e3}

% ARTIFACT: section specifying the size of the artifact master file uploaded
% to the Dagstuhl Research Online Publication Server
\artifactsize{1.7 GB}

% \subparagraph*{Acknowledgements}

% The authors wish to thank \dots

% ARTIFACT: optional appendix
%\appendix

%\section{My Appendix}

% Add here any further material you would like to include. For instance, if the artifact is itself a PDF document, add it here.


% ARTIFACT: include here any additional references, if needed...

%% Either use bibtex (recommended), but commented out in this sample

%\bibliography{dummybib}

%% .. or use the thebibliography environment explicitely

% \nocite{Simpson}

% \begin{thebibliography}{50}
% \bibitem{Simpson} Homer J. Simpson. \textsl{Mmmmm...donuts}. Evergreen Terrace Printing Co., Springfield, Somewhere, USA, 1998
% \end{thebibliography}


\end{document}
