
\documentclass[a4paper,UKenglish]{darts}

\usepackage{microtype}

\bibliographystyle{plainurl}

\newenvironment{scope}{\section{Scope}}{}
\newenvironment{content}{\section{Content}}{}
\newenvironment{getting}{\section{Getting the artifact} The artifact 
endorsed by the Artifact Evaluation Committee is available free of 
charge on the Dagstuhl Research Online Publication Server (DROPS).}{}
\newenvironment{platforms}{\section{Tested platforms}}{}
\newcommand{\license}[1]{{\section{License}#1}}
\newcommand{\mdsum}[1]{{\section{MD5 sum of the artifact}#1}}
\newcommand{\artifactsize}[1]{{\section{Size of the artifact}#1}}

\title{Interprocedural Specialization of Higher-Order Dynamic Languages Without Static Analysis (Artifact)\footnote{This artifact is a companion of the paper:  Baptiste Saleil and Marc Feeley, ``Interprocedural Specialization of Higher-Order Dynamic Languages Without Static Analysis'', Proceedings of the 31st European Conference on Object-Oriented Programming (ECOOP 2017), June 18-23, 2017, Barcelona, Spain.}}
\titlerunning{Interprocedural Specialization of Higher-Order Dynamic Languages Without Static Analysis (Artifact)}

\author[1]{Baptiste Saleil}
\author[2]{Marc Feeley}

\affil[1]{Universit\'e de Montr\'eal\\
  Montreal, Quebec, Canada\\
  \texttt{baptiste.saleil@umontreal.ca}}
\affil[2]{Universit\'e de Montr\'eal\\
  Montreal, Quebec, Canada\\
  \texttt{feeley@iro.umontreal.ca}}

\authorrunning{B., Saleil and M., Feeley}
\Copyright{Baptiste Saleil and Marc Feeley}

\subjclass{D.3.4 Processors}
\keywords{Just-in-time compilation, Interprocedural optimization, Dynamic language, Higher-order function, Scheme}

\Volume{3}
\Issue{2}
\Article{14}
\RelatedConference{European Conference on Object-Oriented Programming (ECOOP 2017), June 18-23, 2017, Barcelona, Spain}

\begin{document}

\maketitle

\begin{abstract}
    This artifact is based on LC, a research oriented JIT compiler for Scheme.
    The compiler is extended to allow interprocedural, type based, code specialization
    using the technique and its implementation presented in the paper.
    Because the technique is directly implemented in LC, the package contains the build
    of the compiler used for our experiments.
    To support repeatability, the artifact allows the user to
    easily extract the data presented in the paper such as the number of executed type
    checks or the generated code size.
    The user can repeat the experiments using a set of standard benchmarks as well as its own programs.
    Instructions for building the compiler from scratch are also provided.
\end{abstract}

\begin{scope}
    The artifact is designed to support repeatability of all the experiments presented in the paper.
    Users can obtain the result from the metrics used for the experiments such as the number of executed type checks,
    the size of the generated machine code or the execution time using a set of standard benchmarks as well
    as their own Scheme programs.
    A tool allowing to automatically gather the data presented in the paper is also included in the artifact.
\end{scope}

\begin{content}

    This artifact contains a build of the LC compiler used for the experiments presented in the paper.
    The archive contains a virtual appliance packaged using the Open Virtualization Format (OVF) and
    the artifact documentation.
    This appliance can be imported and started using the virtualization software \textit{VirtualBox}.

    When the system is booted, the user can find the artifact in a folder named \texttt{artifact} on the desktop.
    To facilitate the use of the artifact, two windows are opened on boot:
    \begin{itemize}
        \item A file explorer window showing the content of the \texttt{artifact} folder
        \item A terminal window opened in the \texttt{artifact} folder
    \end{itemize}

    The \texttt{artifact} folder contains:

    \begin{itemize}
        \item The artifact documentation
        \item The paper
        \item The sources of the compiler
        \item The benchmarks used for the experiments
    \end{itemize}
\end{content}

\begin{getting}
    The latest version of the compiler is available on Github:
    {\url https://github.com/bsaleil/lc}
\end{getting}

\begin{platforms}
    Because the compiler targets x86-64 assembly,
    the artifact must be executed on a platform using a x86-64 CPU supporting SSE extensions.
    The artifact is known to work on any operating system satisfying this condition and running Oracle VirtualBox
    version~5 ({\url https://www.virtualbox.org/}) with at least 8~GB of free
    space on disk and at least 3~GB of free space in RAM.
\end{platforms}

\license{BSD-3-Clause ({\url https://opensource.org/licenses/BSD-3-Clause})}

\mdsum{7a431b74c95241dc09445eeed4790ca0}

\artifactsize{2.8 GB}

\end{document}
