% This is a template for producing artifact descriptions associated with ECOOP 2017 papers
%
% Following is the notice from Camil Demetrescu's ECOOP 2016 template on which this
% template is based:
% This is a template for producing artifact descriptions associated with ECOOP papers
% Adapted from the standard LIPIcs template by Camil Demetrescu
% See lipics-manual.pdf for further information.
% April 22, 2015

\documentclass[a4paper,UKenglish]{darts}
% for A4 paper format use option "a4paper", for US-letter use option "letterpaper"
% for british hyphenation rules use option "UKenglish", for american hyphenation rules use option "USenglish"
% for section-numbered lemmas etc., use "numberwithinsect"
 
\usepackage{microtype}%if unwanted, comment out or use option "draft"

%\graphicspath{{./graphics/}}%helpful if your graphic files are in another directory

\bibliographystyle{plainurl}% the recommended bibstyle

% ARTIFACT
\newenvironment{scope}{\section{Scope}}{}
\newenvironment{content}{\section{Content}}{}
\newenvironment{getting}{\section{Getting the artifact} The artifact 
endorsed by the Artifact Evaluation Committee is available free of 
charge on the Dagstuhl Research Online Publication Server (DROPS).}{}
\newenvironment{platforms}{\section{Tested platforms}}{}
\newcommand{\license}[1]{{\section{License}#1}}
\newcommand{\mdsum}[1]{{\section{MD5 sum of the artifact}#1}}
\newcommand{\artifactsize}[1]{{\section{Size of the artifact}#1}}

% Author macros::begin %%%%%%%%%%%%%%%%%%%%%%%%%%%%%%%%%%%%%%%%%%%%%%%%
% ARTIFACT: Please use the same title as the corresponding ECOOP paper and append the text "(Artifact)"
% ARTIFACT: Add as a footnote the reference to the corresponding ECOOP paper
\title{Type Abstraction for Relaxed Noninterference (Artifact)\footnote{This artifact is a companion of the paper:  Raimil Cruz, Tamara Rezk, Bernard Serpette and \'Eric Tanter, ``Type Abstraction for Relaxed Noninterference'', Proceedings of the 31st European Conference on Object-Oriented Programming (ECOOP 2017), June 18-23, 2017, Barcelona, Spain. This work was partially funded by Project Conicyt REDES 140219 ``CEV: Challenges in Practical Electronic Voting''. Raimil Cruz is funded by CONICYT-PCHA/Doctorado Nacional/2014-63140148.}}
\titlerunning{Type Abstraction for Relaxed Noninterference (Artifact)} %optional, in case that the title is too long; the running title should fit into the top page column

% ARTIFACT: Authors may not be exactly the same as the ECOOP paper, e.g., you may want to include authors who contributed to the preparation of the artifact, but not to the ECOOP companion paper
%% Please provide for each author the \author and \affil macro, even when authors have the same affiliation, i.e. for each author there needs to be the  \author and \affil macros
\author[1]{Raimil Cruz}
\author[2]{Tamara Rezk}
\author[3]{Bernard Serpette}
\author[4]{\'Eric Tanter}
\affil[1]{PLEIAD Lab, Computer Science Department (DCC),
University of Chile\\
\textsf{racruz@dcc.uchile.cl}}
\affil[2]{INRIA - Indes Project-Team, Sophia Antipolis, France\\
\textsf{tamara.rezk@inria.fr}}
\affil[3]{INRIA - Indes Project-Team, Sophia Antipolis, France\\
\textsf{bernard.serpette@inria.fr}}
\affil[4]{PLEIAD Lab, Computer Science Department (DCC),
University of Chile\\
\textsf{etanter@dcc.uchile.cl}}
\authorrunning{R. Cruz, T. Rezk, B. Serpette and \'E. Tanter} %mandatory. First: Use abbreviated first/middle names. Second (only in severe cases): Use first author plus 'et. al.'

\Copyright{Raimil Cruz, Tamara Rezk, Bernard Serpette and \'Eric Tanter}%mandatory, please use full first names. LIPIcs license is "CC-BY";  http://creativecommons.org/licenses/by/3.0/

\subjclass{D.4.6 Security and Protection: Information flow controls, D.3.2 Language Classifications: Object-oriented languages}
\keywords{type abstraction, relaxed noninterference, information flow control}
% Author macros::end %%%%%%%%%%%%%%%%%%%%%%%%%%%%%%%%%%%%%%%%%%%%%%%%%

%Editor-only macros:: begin (do not touch as author)%%%%%%%%%%%%%%%%%%%%%%%%%%%%%%%%%%
\Volume{3}
\Issue{2}
\Article{9}
\RelatedConference{European Conference on Object-Oriented Programming (ECOOP 2017), June 18-23, 2017, Barcelona, Spain}
% Editor-only macros::end %%%%%%%%%%%%%%%%%%%%%%%%%%%%%%%%%%%%%%%%%%%%%%%

\begin{document}

\maketitle

\begin{abstract}
This artifact is a web interpreter for the ObSec language defined in the companion paper. 
	ObSec is a simple object-oriented language that supports \emph{type-based declassification}. 
	Type-base declassification exploits the familiar notion of type abstraction to support expressive
declassification policies in a simple and expressive manner. 
 \end{abstract}

% ARTIFACT: please stick to the structure of 7 sections provided below

% ARTIFACT: section on the scope of the artifact (what claims of the paper are intended to be backed by this artifact?)
\begin{scope}
  The artifact is designed to test the semantics of the ObSec language described in companion paper, 
	allowing users to define their own declassification policies.
\end{scope}

% ARTIFACT: section on the contents of the artifact (code, data, etc.)
\begin{content}
  The artifact package includes:
  \begin{itemize}
  \item a Virtual Box Linux machine.
  \item a .zip file including the binaries of the ObSec interpreter.
  \item detailed instructions for using the artifact provided as an {\tt readme.pdf} file.
  \end{itemize}
	
  To simplify the access, we provide an online ObSec Pad at \url{https://pleiad.cl/obsec/} which does not require any installation and is always up-to-date.
	If you want to use an snapshot of the state of the interpreter (at the submission time), then follow the instructions in the {\tt readme.pdf} file to use
	either the Virtual Box machine or the ObSec interpreter binaries.
\end{content} 

% ARTIFACT: section containing links to sites holding the
% latest version of the code/data, if any
\begin{getting}
% leave empty if the artifact is only available on the DROPS server.
% otherwise, provide links to the latest version of the artifact (e.g., on github)
  The latest version of our artifact is available at the PLEIAD Lab website ({\url https://pleiad.cl/research/software/obsec})
\end{getting} 

% ARTIFACT: section specifying the platforms on which the artifact is known to
% work, including requirements beyond the operating system such as large
% amounts of memory or many processor cores
\begin{platforms}
  The artifact is known to work on any platform running Oracle VirtualBox
  version~5.1.18 (\url{https://www.virtualbox.org/}) with at least 9~GB of free
  space on disk and at least 2~GB of free space in RAM.
\end{platforms}

% ARTIFACT: section specifying the license under which the artifact is
% made available
\license{BSD-3 (\url https://opensource.org/licenses/BSD-3-Clause)}

% ARTIFACT: section specifying the md5 sum of the artifact master file
% uploaded to the Dagstuhl Research Online Publication Server, enabling 
% downloaders to check that the file is the expected version and suffered 
% no damage during download.              
\mdsum{6088db75bcf48d9ca75af124b781335e}

% ARTIFACT: section specifying the size of the artifact master file uploaded
% to the Dagstuhl Research Online Publication Server
\artifactsize{2.1 GB}

\subparagraph*{Acknowledgements}
The authors wish to thank to Matias Toro for his feedback during the construction and testing of the artifact, and to the anonymous artifact reviewers
for their valuable feedback to improve the artifact.

% ARTIFACT: optional appendix
%\appendix

%\section{My Appendix}

% Add here any further material you would like to include. For instance, if the artifact is itself a PDF document, add it here.


% ARTIFACT: include here any additional references, if needed...

%% Either use bibtex (recommended), but commented out in this sample

%

%% .. or use the thebibliography environment explicitely
\end{document}
