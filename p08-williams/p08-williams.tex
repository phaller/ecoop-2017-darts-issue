\documentclass[a4paper,UKenglish]{darts}

\usepackage{microtype}

\bibliographystyle{plainurl}% the recommended bibstyle

\newenvironment{scope}{\section{Scope}}{}
\newenvironment{content}{\section{Content}}{}
\newenvironment{getting}{\section{Getting the artifact} The artifact 
endorsed by the Artifact Evaluation Committee is available free of 
charge on the Dagstuhl Research Online Publication Server (DROPS).}{}
\newenvironment{platforms}{\section{Tested platforms}}{}
\newcommand{\license}[1]{{\section{License}#1}}
\newcommand{\mdsum}[1]{{\section{MD5 sum of the artifact}#1}}
\newcommand{\artifactsize}[1]{{\section{Size of the artifact}#1}}

% Author macros::begin %%%%%%%%%%%%%%%%%%%%%%%%%%%%%%%%%%%%%%%%%%%%%%%%
% ARTIFACT: Please use the same title as the corresponding ECOOP paper and append the text "(Artifact)"
% ARTIFACT: Add as a footnote the reference to the corresponding ECOOP paper
\title{Mixed Messages: Measuring Conformance and Non-Interference in
  TypeScript (Artifact)\footnote{This artifact is a companion of the
    paper: Jack Williams, J. Garrett Morris, Philip Wadler, and Jakub
    Zalewski, ``Mixed Messages: Measuring Conformance and
    Non-Interference in TypeScript'', Proceedings of the 31st European
    Conference on Object-Oriented Programming (ECOOP 2017), June
    18-23, 2017, Barcelona, Spain.}}  \titlerunning{Mixed Messages:
  Measuring Conformance and Non-Interference in TypeScript (Artifact)}

\author[1]{Jack Williams\footnote{Core artifact developer.}}
\affil[1]{University of Edinburgh\\
  \texttt{jack.williams@ed.ac.uk}}
\author[2]{J. Garrett Morris}
\affil[2]{University of Edinburgh\\
  \texttt{Garrett.Morris@ed.ac.uk}}
\author[3]{Philip Wadler}
\affil[3]{University of Edinburgh\\
  \texttt{wadler@inf.ed.ac.uk}}
\author[4]{Jakub Zalewski}
\affil[4]{University of Edinburgh\\
  \texttt{jakub.zalewski@ed.ac.uk}}

\authorrunning{J. Williams, J. G. Morris, P. Wadler, and J.
  Zalewski}

\Copyright{Jack Williams, J. Garrett Morris, Philip Wadler, and Jakub
  Zalewski}

\subjclass{D.2.5 [\textit{Software Engineering}]: Testing and Debugging}

\keywords{Gradual Typing, TypeScript, JavaScript, Proxies}
% Author macros::end %%%%%%%%%%%%%%%%%%%%%%%%%%%%%%%%%%%%%%%%%%%%%%%%%

%Editor-only macros:: begin (do not touch as author)%%%%%%%%%%%%%%%%%%%%%%%%%%%%%%%%%%
\Volume{3}
\Issue{2}
\Article{8}
\RelatedConference{European Conference on Object-Oriented Programming (ECOOP 2017), June 18-23, 2017, Barcelona, Spain}
% Editor-only macros::end %%%%%%%%%%%%%%%%%%%%%%%%%%%%%%%%%%%%%%%%%%%%%%%

\begin{document}

\maketitle

\begin{abstract}
  In the paper \textit{Mixed Messages: Measuring Conformance and
    Non-Interference in TypeScript} we present our experiences of
  evaluating gradual typing using our tool TypeScript TPD. The tool,
  based on the polymorphic blame calculus, monitors JavaScript
  libraries and TypeScript clients against the corresponding
  TypeScript definition. Our experiments yield two conclusions. First,
  TypeScript definitions are prone to error. Second, there are serious
  technical concerns with the use of the JavaScript proxy mechanism
  for enforcing contracts. This artifact includes all the libraries we
  tested, their definition files, and the source code of our
  tool. From this, all libraries can be wrapped and tested to
  reproduce the log data that formed our conclusion. All conformance
  errors and examples of interference are documented, and can be
  verified against the generated logs.
 \end{abstract}

 \begin{scope}
   The aim of the artifact is to enable the reproduction of
   conformance errors and examples of interference found through the
   use of TPD. All libraries can be wrapped with TPD and tested to
   produce logs that match the tables presented in the paper.
\end{scope}

\begin{content}
  The artifact is provided as VirtualBox disk image running Ubuntu
  16.04.  Accompanying the disk image is a detail guide on how to
  load, run, and evaluate the artifact, provided as a {\tt README.html}
  file. The main components of the virtual machine are:
  \begin{itemize}
  \item The source code of TypeScript TPD that can be compiled and executed.
  \item The 122 libraries that can be wrapped and tested to generate the logs,
    from which, the examples are taken.
  \item The examples of type-errors and interference collected and recorded.
  \item A small test program that includes examples from the paper. 
  \end{itemize}
  To simplify use of the artifact, all libraries and dependencies are
  pre-installed on the virtual machine. Scripts are provided to
  automate the process of wrapping the libraries, testing the
  libraries, and producing the tables from the paper. A skeleton
  library, client, and definition file are included to allow users to
  experiment with the tool. All stages are described in detail in the
  accompanying \texttt{README.html} file.
  
\end{content} 

\begin{getting}
\end{getting} 

\begin{platforms}
  The artifact is known to work on any platform running Oracle
  VirtualBox version~5 (\url{https://www.virtualbox.org/}) with at
  least 20~GB of free space on disk and at least 4~GB of free space in
  RAM.  Import the image using \texttt{File > Import Appliance}.
\end{platforms}

\license{MIT (\url{http://opensource.org/licenses/MIT})}

\mdsum{b5a1d04103fbd86820b32168ebb0504f}

\artifactsize{5.1 GB}

\subparagraph*{Acknowledgements}
This work was supported by Microsoft Research through its PhD
Scholarship Programme, and by EPSRC grants EP/K034413/1 and
EP/L01503X/1.  The authors wish to thank anonymous reviewers of the
paper and artifact.


\end{document}
